\documentclass[12pt]{article}

\usepackage{headerfooter}
\usepackage{epsfig}
\usepackage{verbatimfiles}
\usepackage{fullpage}
\usepackage{amsmath}

\newcommand{\HRule}{\rule{\linewidth}{.3mm}}

\bibliographystyle{plain}
\begin{document}

\begin{center}
{\bf \Large Uniaxial Return Mapping Algorithm}

{\bf August 21, 2001} 

{\bf Michael H. Scott} 

{\bf PEER, University of California, Berkeley}
\end{center}

This document outlines the return mapping algorithm for a rate-independent
uniaxial material
model with combined isotropic and kinematic hardening. The algorithm and its
derivation are given in Simo and Hughes~\cite{Simo:1998}.

The material parameters are the elastic modulus, $E$, yield stress,
$\sigma_y$, isotropic hardening modulus, $H_{iso}$, and kinematic hardening
modulus, $H_{kin}$. Path dependence is tracked by the plastic strain,
$\varepsilon^p$, internal hardening variable, $\alpha$, and back
stress, $\kappa$.

\section*{Model Description}
The model assumes an elastic stress-strain relationship with elastic modulus $E$.
The onset of plastic flow occurs upon yielding, after which the elastoplastic
tangent is given by $\frac{E(H_{iso}+H_{kin})}{E+H_{iso}+H_{kin}}$, as shown
in figure~\ref{fig:StressStrain}.

\begin{figure}[htpb]
\begin{center}
\input{./fig_files/StressStrain.pstex_t}
\end{center}
\caption{Elastic and elastoplastic tangent}
\label{fig:StressStrain}
\end{figure}

Isotropic hardening can be thought of as an ``expansion'' of the elastic region.
The internal hardening variable, $\alpha$, tracks the growth of the elastic
region.
Kinematic hardening corresponds to a ``translation'' of the elastic region.
The back stress, $\kappa$, is the center of the elastic region. When there is
no kinematic hardening, $\kappa$ is zero.
These two hardening rules are shown in figure~\ref{fig:HardeningBehavior}, and
they can be combined to give mixed hardening behavior.

\begin{figure}[htpb]
\begin{center}
\input{./fig_files/HardeningBehavior.pstex_t}
\end{center}
\caption{Isotropic and kinematic hardening behavior}
\label{fig:HardeningBehavior}
\end{figure}

\section*{Continuum Equations}

\begin{enumerate}
\item{} Elastic stress-strain relationship
\begin{equation}
\sigma = E \left( \varepsilon-\varepsilon^p \right)
\end{equation}

\item{} Flow rule
\begin{equation}
\dot{\varepsilon}^p = \gamma \mbox{sign}\left( \sigma-\kappa \right)
\end{equation}

\item{} Isotropic and kinematic hardening laws
\begin{align}
\dot{\alpha} &= \gamma \\
\dot{\kappa} &= \gamma H_{kin} \mbox{sign}\left( \sigma-\kappa \right)
\end{align}

\item{} Yield condition
\begin{equation}
f(\sigma, \kappa, \alpha) = \left| \sigma-\kappa \right| -
\left( \sigma_y + H_{iso}\alpha \right) \leq 0
\end{equation}

\item{} Kuhn-Tucker complementary conditions
\begin{align}
\gamma &\geq 0 \\
f(\sigma, \kappa, \alpha) &\leq 0 \\
\gamma f(\sigma, \kappa, \alpha) &= 0
\end{align}

\item{} Consistency condition
\begin{equation}
\gamma \dot{f}(\sigma, \kappa, \alpha) = 0 \:\:\:\:\:
\mbox{(if $f(\sigma, \kappa, \alpha)$ = 0)}
\end{equation}

\end{enumerate}

\section*{Return Mapping Algorithm}

\begin{enumerate}

\item{} Committed state at time $t_n$
\begin{equation}
\left\{ \varepsilon_n^p, \alpha_n, \kappa_n \right\}
\end{equation}

\item{} Given trial strain at time $t_{n+1}$,
\begin{equation}
\varepsilon_{n+1} = \varepsilon_n + \Delta\varepsilon_n,
\end{equation}

\noindent determine the corresponding stress, $\sigma_{n+1}$, and
tangent, $D_{n+1}$; proceed to step 3.

\item{} Compute trial stress and test for plastic loading
\begin{align}
\sigma_{n+1}^{trial} &= E \left( \varepsilon_{n+1} - \varepsilon_n^p \right) \\
\xi_{n+1}^{trial} &= \sigma_{n+1}^{trial} - \kappa_n \\
f_{n+1}^{trial} &= \left| \xi_{n+1}^{trial} \right| -
                  \left( \sigma_y + H_{iso}\alpha_n \right)
\end{align}

\begin{center}
If $f_{n+1} \leq 0$, this is an elastic step; set
$\sigma_{n+1} = \sigma_{n+1}^{trial}, D_{n+1} = E$, and exit. \\
Else, this is a plastic step; proceed to step 4.
\end{center}

\item{} Return mapping
\begin{align}
\Delta\gamma &= \frac{f_{n+1}^{trial}}{E+H_{iso}+H_{kin}} \\
\sigma_{n+1} &= \sigma_{n+1}^{trial} - \Delta\gamma E \mbox{sign}(\xi_{n+1}^{trial}) \\
\varepsilon_{n+1}^p &= \varepsilon_n^p + \Delta\gamma \mbox{sign}(\xi_{n+1}^{trial}) \\
\kappa_{n+1} &= \kappa_n + \Delta\gamma H_{kin} \mbox{sign}(\xi_{n+1}^{trial}) \\
\alpha_{n+1} &= \alpha_n + \Delta\gamma \\
D_{n+1} &= \frac{E(H_{iso}+H_{kin})}{E+H_{iso}+H_{kin}}
\end{align}

\end{enumerate}

\begin{thebibliography}{99}
\bibitem{Simo:1998} J.C. Simo and T.J.R. Hughes,
\emph{Computational Inelasticity}. Springer-Verlag, 1998.
\end{thebibliography}

\end{document}
