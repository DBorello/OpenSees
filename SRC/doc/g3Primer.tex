\documentclass[12pt]{article}

\usepackage{headerfooter}
\usepackage{epsf}
\usepackage{epsfig}
\usepackage{rotating}
\usepackage{subfigure}
\usepackage{multirow}
\usepackage{verbatimfiles}
\usepackage{fullpage}
\newcommand{\HRule}{\rule{\linewidth}{.3mm}}

\begin{document}
\bibliographystyle{plain}

\begin{center}
{\bf \Large The OpenSees Command Language Primer} 

{\bf Version 1.0 - Preliminary Draft} 

{\bf  September 15, 2000} 

{\bf  Frank McKenna and Gregory L. Fenves} 

{\bf  PEER, University of California at Berkeley}
\end{center}

\vspace{.2in}
\section{Introduction}
This document is intended to outline the rudimentary commands
currently available with opensees. opensees is an interpreter being developed for
use with OpenSees. OpenSees is an object-oriented framework under construction for
finite element analysis. OpenSees's intended users are in the research
community. A key feature of OpenSees is the interchangeability of components
and the ability to integrate existing libraries and new components
into the framework (not just new elements classes) without the need to
change the existing code. Core components, that is the abstract base
classes, define the minimal interface (minimal to make adding new
component classes easier but large enough to ensure all that is
required can be accommodated). 

OpenSees is comprised of a set of modules to perform creation of the finite 
element model, specification of an analysis procedure, specification 
of quantities to be monitored during the analysis, and the output of
results. In each finite element analysis, an analyst constructs 4 main
types of objects, as shown in figure~\ref{main}:


\begin{figure}[htpb]
\begin{center}
\leavevmode
\hbox{%
%\epsfxsize=6.0in
%\epsfysize=4.2in
\epsffile{./Essential.eps}}
\end{center}
\caption{Main Objects in an Analysis}
\label{main}
\end{figure}

\begin{enumerate}
\item {\bf ModelBuilder}: As in any finite element analysis, the analyst's first
step is to subdivide the body under study into elements and nodes,
to define loads acting on the elements and nodes, and to define constraints
acting on the nodes. The ModelBuilder object is that object in the program
responsible for building the Element, Node, LoadPattern, TimeSeries,
Load and Constraint objects.  

\item {\bf Domain}: The Domain object is  responsible for
storing the objects created by the ModelBuilder object and for providing the
Analysis and Recorder objects access to these objects.

\item {\bf Analysis}: Once the analyst has defined the model, the next step
is to define the analysis that is to be performed on the model. This
may vary from a simple static linear analysis to a transient
non-linear analysis. The Analysis object is responsible for performing
the analysis. In OpenSees each Analysis object is composed of several component
objects, which define how the analysis is performed. The component
classes consist of the following: { SolutionAlgorithm}, {
Integrator}, { ConstraintHandler}, { DOF\_Numberer}, {
SystemOfEqn}, { Solver}, and { AnalysisModel}. 

\item {\bf Recorder}: Once the model and analysis have been defined, the
analyst has the option of specifying what is to be
monitored during the analysis. This, for example, could be the
displacement history at a node in a transient analysis or the entire
state of the model at each step in the solution procedure. Several
Recorder objects are created by the analyst to monitor the analysis.
\end{enumerate}

The main abstractions of OpenSees will be explained using opensees. opensees is an
interpreter for openseesTcl, an extension of the Tcl scripting language.
Tcl is a string based procedural command language which allows
substitution, loops, mathematical expressions, and procedures. 
opensees adds commands to Tcl for finite element analysis.
Each of these commands is associated (bound) with a C++
procedure that is provided. It is this procedure that is called upon
by the interpreter to parse the command. In this document we
outline only those commands which have been added to Tcl by
opensees. Following the outline of the new commands, an example 
openseesTcl script is presented which performs an analysis of a simple two
dimensional truss model. 

\section {Tcl Basics}
The basic syntax for a Tcl command is    

{\sf\small
\begin{verbatim}
   command arg1 arg2 aropensees ...
\end{verbatim}
}

\noindent where command is the name of the Tcl command or is a user defined
procedure and arg1 arg2 ... are the arguments for the command. Tcl allows
any argument to be a nested command:

{\sf\small
\begin{verbatim}
   command [nested command 1] [nested command 2] ...
\end{verbatim}
}

\noindent where the [ ] are used to delimit the nested commands. The Tcl
interpreter will first evaluate the nested commands and will evaluate
the outer command with the result of the nested commands.

The most basic command in Tcl is the set coommand:

{\sf\small
\begin{verbatim}
   set variable value
\end{verbatim}
}

\noindent which takes two arguments, the variables name and the value it is to
be assigned. Value may be a string or number (in Tcl everything is
treated as a string). To obtain the value of a variable, variable
substitution, the $\$$ operator is used.

To evaluate mathematical expressions the expr command is used:

{\sf\small
\begin{verbatim}
   expr expression
\end{verbatim}
}

\noindent where the expression may be any valid mathematical
expression used in the C programming language. Tcl allows variable
substitution in the expression.

Double quotes and braces can be used to group strings into one
argument for a command. The difference is that quotes allow
substitution to occur in the group, where as braces do not, for example:

{\sf\small
\begin{verbatim}
	set a 5
	-> 5
	puts "a is $a"
	-> a is 5
	puts {a is $a}
	-> a is $a
\end{verbatim}
}

Procedures are defined using the command proc:

{\sf\small
\begin{verbatim}
	proc name args body
\end{verbatim}
}

\noindent where name is the name of the Tcl procedure created, args is the
procedure arguments and body the body of the procedure, for example:

{\sf\small
\begin{verbatim}
	set a 5
	proc sum {arg1 arg2} {
		return [expr $arg1 + $arg2]
	}
	sum $a $a
	->10
\end{verbatim}
}

Tcl also allows for loops and conditional evaluation. For more
details see 'Practical Programming in Tcl and Tk' by Brent B. Welch.

\section{Notation}
For the rest of this document the following notation will be
used. Input values are a string unless terminated by a $?$, in which
case an 
integer or floating point number is to be provided. Optional values
are identified in enclosing $<$ $>$ braces. When specifying a quantity
of x values are required, the command line contains (x values?). An
arbitrary number of input values is indicated with the dotdotdot
notation, i.e. value1? value2? ..., where an arbitrary number of
values can be provided.


\section{The model Command}
{\sf\small
\begin{verbatim}
   model modelBuilderType <specific model builder args>
\end{verbatim}
}

\noindent The model command has at least one argument which identifies
the type of ModelBuilder object to be constructed. Currently there is
only one type of ModelBuilder accepted, that of type BasicBuilder. 

{\sf\small
\begin{verbatim}
   model BasicBuilder -ndm ndm? <-ndf ndf?>
\end{verbatim}
}

\noindent The command for constructing a BasicBuilder object contains
additional arguments. The string -ndm followed by an integer
defining the dimension of the problem, i.e. 1,2 or 3-d. By default the
number of degrees-of-freedom at a node (ndf) depend on the value of
ndm (ndm=1, ndf=1; ndm=2, ndf=3; ndm=3, ndf=6). An optional
string -ndf followed by an integer defining the number of degrees
associated with each node can also be specified if the analyst should
require degrees of freedom different from the defaults.

The construction of the BasicBuilder object adds additional commands
to the opensees language. These additional commands allow for the
construction of Nodes, Elements, LoadPatterns, TimeSeries, Loads and
Constraints. The additional commands are as follows:

\subsection{The node Command}
{\sf\small
\begin{verbatim}
   node nodeTag? (ndm coordinates?) <-mass (ndf values?)>
\end{verbatim}
}

The node command is used to construct a Node object. The first
argument to the node command defines the integer tag that uniquely
identifies the node object among all other nodes in the model. Following the tag
argument, ndm nodal coordinates must be provided to define the spatial
location of the Node. An optional string -mass accompanied by ndf
mass terms following the specification of the coordinates allows the
analyst the option of associating nodal mass with the Node.

\subsection{The mass Command}
{\sf\small
\begin{verbatim}
   mass nodeTag? (ndf values?)
\end{verbatim}
}

The mass command is used to set the mass at a node. The first
argument to the node command defines the integer tag that uniquely
identifies the node for which the mass will be set. Following the tag
argument, ndf mass terms are specified, where ndf is the number of
degrees of freedom per node in the model.

\subsection{The uniaxialMaterial Command}
{\sf\small
\begin{verbatim}
   uniaxialMaterial materialType <specific material args>
\end{verbatim}
}

The uniaxialMaterial command is used to construct a UniaxialMaterial object. 
UniaxialMaterial objects represent uniaxial stress-strain (or force-deformation)
relationships. The command has at least one argument, the string materialType, which
identifies the type of material being constructed. Currently the following types 
are permitted: Elastic, ElasticPP, ElasticPPGap, Parallel, Series, Hardening,
Steel01, Concrete01, Hysteretic, and Viscous. The commands for specifying each type are
as outlined below.

The valid queries to any uniaxial material when creating an ElementRecorder
are 'strain', 'stress', and 'tangent'.

\subsubsection{Elastic Material}
{\sf\small
\begin{verbatim}
   uniaxialMaterial Elastic matTag? E? <eta?>
\end{verbatim}
}

\noindent To construct an elastic uniaxial material with a tangent of E
and optional damping tangent of eta. The argument matTag is used to
uniquely identify this UniaxialMaterial object among UniaxialMaterial objects
in the BasicBuilder object.

\subsubsection{Elastic-Perfectly Plastic Material}
{\sf\small
\begin{verbatim}
   uniaxialMaterial ElasticPP matTag? E? ep?
\end{verbatim}
}

\noindent To construct an elastic perfectly plastic uniaxial material
with an elastic tangent of E which
reaches the plastic state at a strain of ep. The argument matTag is used to
uniquely identify this UniaxialMaterial object among UniaxialMaterial objects
in the BasicBuilder object.

\subsubsection{Elastic-Perfectly Plastic Gap Material}
{\sf\small
\begin{verbatim}
   uniaxialMaterial ElasticPPGap matTag? E? fy? gap?
\end{verbatim}
}

\noindent To construct an elastic perfectly plastic gap uniaxial material
with an elastic tangent of E, which reaches the plastic state at a stress 
of fy. The initial gap of the model is given by the argument gap. The argument
matTag is used to uniquely identify this UniaxialMaterial object among
UniaxialMaterial objects in the BasicBuilder object.

\subsubsection{Parallel Material}
{\sf\small
\begin{verbatim}
   uniaxialMaterial Parallel matTag? tag1? tag2? ... <-min min?> <-max max?>
\end{verbatim}
}

\noindent To construct a parallel material model made up of an
arbitrary number of previously constructed UniaxialMaterial objects, 
which are identified by the tags tag1 tag2 ... . In a parallel model,
strains are equal and stresses and tangents are additive. Specification of
minimum and maximum failure strains through the -min and -max switches
is optional. The argument matTag is used to
uniquely identify this UniaxialMaterial object among UniaxialMaterial objects
in the BasicBuilder object.

\subsubsection{Series Material}
{\sf\small
\begin{verbatim}
   uniaxialMaterial Series matTag? tag1? tag2? ... 
\end{verbatim}
}

\noindent To construct a series material model made up of an
arbitrary number of previously constructed UniaxialMaterial objects, 
which are identified by the tags tag1 tag2 ... . In a series model,
stresses are equal and strains and flexibilities are additive. The argument matTag 
is used to uniquely identify this UniaxialMaterial object among UniaxialMaterial objects
in the BasicBuilder object.

\subsubsection{Hardening Material}
{\sf\small
\begin{verbatim}
   uniaxialMaterial Hardening matTag? E? sigmaY? H_iso? H_kin?
\end{verbatim}
}

\noindent To construct a uniaxial material model with combined linear kinematic
and isotropic hardening.
The model contains a yield strength of fy, an initial elastic
tangent of E, an isotropic hardening modulus of H\_iso, and a kinematic hardening
modulus of H\_kin. The argument matTag is used to
uniquely identify this UniaxialMaterial object among UniaxialMaterial objects
in the BasicBuilder object.

\subsubsection{Steel01 Material}
{\sf\small
\begin{verbatim}
   uniaxialMaterial Steel01 matTag? fy? E0? b? <a1? a2? a3? a4?> 
                         <-min min?> <-max max?>
\end{verbatim}
}

\noindent To construct a uniaxial bilinear steel model with kinematic hardening and
optional isotropic hardening described by a non-linear evolution equation.
The model contains a yield strength of fy, an initial elastic
tangent of E0, and a hardening ratio of b. The optional parameters a1, a2, a3,
and a4 control the amount of isotropic hardening (default values are
provided). Specification of minimum and maximum failure strains
through the -min and -max switches is optional and must appear after
the specification of the isoptropic hardening parameters, if present. The argument 
matTag is used to uniquely identify this UniaxialMaterial object among UniaxialMaterial 
objects in the BasicBuilder object.

\subsubsection{Concrete01 Material}
{\sf\small
\begin{verbatim}
   uniaxialMaterial Concrete01 matTag? fpc? epsc0? fpcu? epscu? <-min min?> <-max max?>
\end{verbatim}
}

\noindent To construct a uniaxial Kent-Scott-Park concrete model with degraded 
linear unloading/reloading according to the work of Karsan-Jirsa and no
strength in tension. The model contains a compressive strength of fpc,
a strain at the compressive strength of epsc0, a crushing strength of
fpcu, and a strain at the crushing strength of epscu. Compressive
concrete parameters should be input as negative values for
this model. Specification of minimum and maximum
failure strains through the -min and -max switches is optional. The argument 
matTag is used to uniquely identify this UniaxialMaterial object among 
UniaxialMaterial objects in the BasicBuilder object.

\subsubsection{Hysteretic Material}
{\sf\small
\begin{verbatim}
   uniaxialMaterial Hysteretic matTag? s1p? e1p? s2p? e2p? <s3p? e3p?>
       s1n? e1n? s2n? e2n? <s3n? e3n?> pinchX? pinchY? damage1? damage2? <beta?>
\end{verbatim}
}

\noindent To construct a bilinear hysteretic model with pinching of force and
deformation, damage due to ductility and energy, and degraded unloading 
stiffness based on ductility. Points on the backbone are specifed by the
arguments s1p, e1p, etc., where s indicates force, e indicates deformation,
1 is the first point on the backbone (yield), 2 the second point, and 3 indicates
an optional third point for a trilinear backbone. p indicates positive backbone
points, and n negative backbone points. Note that negative backbone points
should be entered as negative numeric values. The pinching factors pinchX and
pinchY indicate the amount of pinching of deformation and force, respectively,
during reloading. Note that setting both pinchX and pinchY to zero gives Clough type 
reloading behavior. The damage factors damage1 and damage2 are for damage due
to ductility, $D_1(\mu-1)$, and energy, $D_2(\frac{E_i}{E_{ult}})$, respectively. 
The optional parameter beta is a power used to determine degraded unloading stiffness
based on ductility, $\mu^{-\beta}$.

% \subsubsection{Viscous Material}
% {\sf\small
% \begin{verbatim}
%    uniaxialMaterial Viscous matTag? C? alpha?
% \end{verbatim}
% }

% \noindent To construct a uniaxial material model with a non-linear elastic
% force-deformation relation given by $F = Cu^\alpha$, where C and $\alpha$ are constant.
% The argument matTag is used to uniquely identify this UniaxialMaterial object 
% among UniaxialMaterial objects in the BasicBuilder object.

\subsection{The nDMaterial Command}
{\sf\small
\begin{verbatim}
   nDMaterial materialType <specific material args>
\end{verbatim}
}

The nDMaterial command is used to construct an NDMaterial object. NDMaterial
objects represent stress-strain relationships at the integration points of
continuum elements. The command has at least one argument, the material type. 
Currently the following types are permitted: ElasticIsotropic. The commands for 
specifying each type are as outlined below. 

The valid queries to any ND material when creating an ElementRecorder
are 'strain', 'stress', and 'tangent'.

\subsubsection{Elastic Isotropic Material}
{\sf\small
\begin{verbatim}
   nDMaterial ElasticIsotropic matTag? E? v?
\end{verbatim}
}

To construct an ElasticIsotropic material object with elastic modulus
E and Poisson ratio v. The argument matTag is used to
uniquely identify this NDMaterial object among NDMaterial objects
in the BasicBuilder object.

The material formulations for the ElasticIsotropic object are 'PlaneStrain2D' and
'PlaneStress2D'.  These are the valid strings that can be passed to the quad
element for the type parameter.  Other types will be added as more continuum elements
become available, e.g. 'Axisymmetric'.

\subsection{The section Command}
{\sf\small
\begin{verbatim}
   section sectionType <specific section args>
\end{verbatim}
}

The section command is used to construct a Section object. Section objects
represent force-deformation relationships at beam-column sample points. The 
command has at least one argument, the section type. Currently the
following types are permitted: Elastic, Generic1d, GenericNd, Aggregator,
and Fiber. The commands for specifying each type are
as outlined below.

The valid queries to any section when creating an ElementRecorder
are 'force' and 'deformation'.

\subsubsection{Elastic Section}
{\sf\small
\begin{verbatim}
   section Elastic secTag? E? A? Iz? <Iy? G? J?>
\end{verbatim}
}

To construct an ElasticSection object with axial stiffness EA and 
bending stiffness EIz about the section local z-axis.  The values 
for EIy and GJ, the bending stiffness about the section local
y-axis and the section torsional stiffness, respectively, are optional
as needed. The argument secTag is used to uniquely identify this 
Section object among Section objects in the BasicBuilder object.

\subsubsection{Generic1d Section}
{\sf\small
\begin{verbatim}
   section Generic1d secTag? matTag? code
\end{verbatim}
}

To construct a GenericSection1d object which uses a previously defined
UniaxialMaterial object, identifed by the argument matTag, to represent
a single section force-deformation response quantity. The argument code indicates
the force-deformation quantity to be modeled by this section object.
Values for code are given in the figure below. The 
argument secTag is used to uniquely identify this Section object among 
Section objects in the BasicBuilder object. 

\begin{figure}[htpb]
\begin{center}
\begin{tabular}{||r|l||} \hline
 P & Axial force-deformation \\
 Mz & Moment-curvature about section local z-axis \\
 Vy & Shear force-deformation along section local y-axis \\
 My & Moment-curvature about section local y-axis \\
 Vz & Shear force-deformation along section local z-axis \\
 T & Torsion force-deformation \\ \hline
\end{tabular}
\caption{Section force-deformation codes}
\label{sectionCodes}
\end{center}
\end{figure}

\subsubsection{GenericNd Section}
{\sf\small
\begin{verbatim}
   section GenericNd secTag? NDTag? code1 code2 ...
\end{verbatim}
}

To construct a GenericSectionNd object which uses a previously defined
NDMaterial object, identifed by the argument NDTag, to represent
a multiple section force-deformation response quantities. The arguments code1, 
code2, ... indicate the force-deformation quantities to be modeled by this section object.
The number of code arguments must match the order of the NDMaterial object.  code1
is mapped to the first stress-strain relation of the NDMaterial, code2 to the second, etc.
Values for code are given in Figure~\ref{sectionCodes}. The argument secTag is used to 
uniquely identify this Section object among Section objects in the BasicBuilder object.

\subsubsection{Section Aggregator}
{\sf\small
\begin{verbatim}
   section Aggregator secTag? matTag1? code1 matTag2? code2 ... <-section sectionTag?>
\end{verbatim}
}

To construct a SectionAggregator object which groups multiple previously defined
UniaxialMaterial objects, represented by the arguments matTag1 code1 matTag2 code2 ...,
into a single section force-deformation model.  The optional -section switch is used
to specify a previously defined Section object, identified by the argument sectionTag,
to which these UniaxialMaterial objects may be added to recursively define a new Section object.
The UniaxialMaterial objects aggregated in this Section object are uncoupled from each
other as well as from the Section object represented by sectionTag, if present.
Values for code are given in Figure~\ref{sectionCodes}. The argument secTag is used to 
uniquely identify this Section object among Section objects in the BasicBuilder object.

\subsubsection{Fiber Section}
{\sf\small
\begin{verbatim}
   section Fiber secTag? {
      patch <patch arguments>
      layer <layer arguments>
   }
\end{verbatim}
}


\noindent To construct a FiberSection object composed of Fiber objects.
The available Fiber objects are UniaxialFiber2d/3d, which enforce the
Bernoulli beam assumption. The secTag is used to uniquely identify the 
section object among section objects in the BasicBuilder object. A fiber 
section has a general geometric configuration formed by subregions
of simpler, regular shapes (e.g. quadrilateral, circular and triangular
regions) called patches. In addition, layers of reinforcement bars 
can also be specified. The subcommands patch and
layer are used to define the discretization of the section into
fibers, and are described below. 
In these subcommands, the geometric parameters are defined
with respect to a planar local coordinate system (y,z). 

\paragraph{The patch Command}

{\sf\small
\begin{verbatim}
   patch patchType <specific patch args>
\end{verbatim}
}

The patch command is used to construct a Patch object. 
The command has at least one argument, the patch type. Currently 
the following types are permitted: quad and circ. The fibers
generated by the patch commands are UniaxialFiber2d/3d, depending on
the dimension of the problem. The commands
for specifying each type are described below:

{\sf\small
\begin{verbatim}
   patch quad matTag? numSubdivIJ? numSubdivJK? yVertI? zVertI? yVertJ? zVertJ? 
                                                yVertK? zVertK? yVertL? zVertL?

\end{verbatim}
}

\noindent To construct a patch with a quadrilateral shape. The
argument matTag is an integer used to identify the associated
UniaxialMaterial (which must be defined previously). The geometry of 
the patch is defined by four vertices I, J, K and L, as illustrated
in figure~\ref{quadPatch}. The arguments numSubdivIJ and numSubdivJK are integers
that specify the number of subdivisions (fibers) along the IJ and JK 
directions, respectively. The last arguments yVertI, zVertI, yVertJ,
zVertJ, yVertK, zVertK, yVertL, and  zVertL
are the coordinates y and z of each of the four vertices specified in sequence
(counter-clockwise).
 


\begin{figure}[htpb]
\begin{center}
\leavevmode
\hbox{%
%\epsfxsize=6.0in
%\epsfysize=4.2in
\epsffile{./quadPatch.eps}}
\end{center}
\caption{Quadrilateral patch}
\label{quadPatch}
\end{figure}


{\sf\small
\begin{verbatim}
   patch circ matTag? numSubdivCirc? numSubdivRad? yCenter? zCenter? 
                                   intRad? extRad? startAng? endAng?
\end{verbatim}
}

\noindent To construct a patch with a circular shape. The
argument matTag is an integer used to identify the associated
UniaxialMaterial (which must be defined previously). 
The arguments numSubdivCirc and numSubdivRad are integers
that specify the number of subdivisions (fibers) along the
circumferential and radial directions, respectively. 
The geometry of the patch is defined by its center position
(yCenter and zCenter), the internal and external radius 
(intRad and extRad), and the starting and ending angle (startAng 
and endAng), according to figure~\ref{circPatch}. 



\begin{figure}[htpb]
\begin{center}
\leavevmode
\hbox{%
%\epsfxsize=6.0in
%\epsfysize=4.2in
\epsffile{./circPatch.eps}}
\end{center}
\caption{Circular patch}
\label{circPatch}
\end{figure}

\paragraph{The layer Command}

{\sf\small
\begin{verbatim}
   layer layerType <specific patch args>
\end{verbatim}
}

The layer command is used to construct a Layer object. 
The command has at least one argument, the layer type. Currently 
the following types are available: straight and circ. The fibers
generated by the layer commands are UniaxialFiber2d/3d, depending on
the dimension of the problem. The commands
for specifying each type are described below:

{\sf\small
\begin{verbatim}
   layer straight matTag? numReinfBars? reinfBarArea? yStartPt? zStartPt? 
                                                      yEndPt? zEndPt?
\end{verbatim}
}

\noindent To construct a straight layer of reinforcing bars. The
argument matTag is an integer used to identify the associated
UniaxialMaterial (which must be defined previously). The argument 
numReinfBars is an integer that specifies the number of reinforcing 
bars, with area reinfBarArea, along the layer. The last arguments 
yStartPt, zStartPt, yEndPt and zEndPt are the coordinates y and z of 
the starting and ending points of the
reinforcing layer, as represented in figure~\ref{straightLayer}.


\begin{figure}[htpb]
\begin{center}
\leavevmode
\hbox{%
%\epsfxsize=6.0in
%\epsfysize=4.2in
\epsffile{./straightLayer.eps}}
\end{center}
\caption{Straight reinforcing layer}
\label{straightLayer}
\end{figure}




{\sf\small
\begin{verbatim}
   layer circ matTag? numReinfBars? reinfBarArea? 
         yCenter? zCenter? radius? startAng? endAng?
\end{verbatim}
}

\noindent To construct a layer with a circular shape. The
argument matTag is an integer used to identify the associated
material (which must be defined previously). The argument 
numReinfBars is an integer that specifies the number of reinforcing 
bars, with area reinfBarArea, along the layer. 
The geometry of the patch is defined by its center position
(yCenter and zCenter), its radius, and the starting and ending angle 
(startAng and endAng), as shown in figure~\ref{circLayer}. 


\begin{figure}[htpb]
\begin{center}
\leavevmode
\hbox{%
%\epsfxsize=6.0in
%\epsfysize=4.2in
\epsffile{./circLayer.eps}}
\end{center}
\caption{Circular reinforcing layer}
\label{circLayer}
\end{figure}


\subsection{The geomTransf Command}
{\sf\small
\begin{verbatim}
   geomTransf transfType <specific transf args>
\end{verbatim}
}

The geomTransf command is used to construct a CrdTransf object. A CrdTransf
object transforms beam element stiffness and resisting force from the 
basic system to the global coordinate system. The command has at least one 
argument, the transformation type. Currently the following types are 
permitted: Linear and LinearWithPDelta.  The commands for specifying each 
type are as outlined below: 

\subsubsection{The Linear Transformation}
{\sf\small
\begin{verbatim}
   geomTransf Linear transfTag? <-jntOffset dXi? dYi? dXj? dYj?>
   geomTransf Linear transfTag? vecxzX? vecxzY? vecxzZ?
                     <-jntOffset dXi? dYi? dZi? dXj? dYj? dZj?>
\end{verbatim}
}

To construct a LinearCrdTransf object which performs a linear geometric
transformation of beam stiffness and resisting force from the basic system
to the global coordinate system.  For the three dimensional problem, additional arguments
need to be specified. Optional rigid joint offsets can be specified
with the -jntOffset switch.  The joint offset values dXi, dYi, dZi and dXj, dYj, dZj 
are absolute offsets with respect to the global coordinate system from element end 
nodes I and J, respectively.  The rigid joint offset arguments depend on the dimension of the
current model.  The argument transfTag is used to uniquely identify this 
CrdTransf object among CrdTransf objects in the BasicBuilder object.

The element coordinate system is specified,
according to figure~\ref{localAxis}, as follows: the x axis is the axis connecting the two
element nodes; the y and z axis are then defined using a vector that
lies on the local xz plane (the components of this vector,  
vecxzX, vecxzY, vecxzZ are specified with respect to the global coordinate system X,Y,Z).
The section is attached to the element such that the (y,z) 
coordinate system used to specify the section corresponds to the (y,z) axes 
of the element. 

\subsubsection{The Linear P-Delta Transformation}
{\sf\small
\begin{verbatim}
   geomTransf LinearWithPDelta transfTag? <-jntOffset dXi? dYi? dXj? dYj?>
   geomTransf LinearWithPDelta transfTag? vecxzX? vecxzY? vecxzZ? 
                               <-jntOffset dXi? dYi? dZi? dXj? dYj? dZj?>
\end{verbatim}
}

To construct a LinearCrdTransf object which performs a linear geometric
transformation of beam stiffness and resisting force from the basic system
to the global coordinate system considering 
second order P-Delta effects. The arguments vecxzX, vecxzY, and vecxzZ are as defined
for the Linear transformation above. Optional rigid joint offsets can be specified
with the -jntOffset switch.  The joint offset values dXi, dYi, dZi and dXj, dYj, dZj 
are absolute offsets with respect to the global coordinate system from element end 
nodes I and J, respectively.  The rigid joint offset arguments depend on the dimension of the
current model.  The argument transfTag is used to uniquely identify this 
CrdTransf object among CrdTransf objects in the BasicBuilder object.

\begin{figure}[htpb]
\begin{center}
\leavevmode
\hbox{%
%\epsfxsize=6.0in
%\epsfysize=4.2in
\epsffile{./localAxis.eps}}
\end{center}
\caption{Definition of the local coordinate system}
\label{localAxis}
\end{figure}

\subsection{The element Command}
{\sf\small
\begin{verbatim}
   element eleType <specific element type args>
\end{verbatim}
}

The element command is used to construct an Element object. The
command has at least one argument, the element type. Currently the
following types are permitted: truss, elasticBeamColumn,
nonlinearBeamColumn, beamWithHinges, zeroLength, and quad.
The commands for specifying each type are as outlined below: 

\subsubsection{The Truss Element}
{\sf\small
\begin{verbatim}
   element truss eleTag? iNode? jNode? A? matTag?
   element truss eleTag? iNode? jNode? secTag?
\end{verbatim}
}

\noindent There are two ways to construct a truss object. One way is
to specify an area and a UniaxialMaterial identifier, the other to specify a
Section identifier. To construct a Truss object with an area and a
UniaxialMaterial, the analyst specifies the eleTag, which uniquely defines
this truss element among all other elements in the domain, the two end
nodes, iNode and jNode, the truss bar area A and the tag of the
associated UniaxialMaterial, matTag. Note that the material must have already
been added to the BasicBuilder object. To construct a Truss object
with a Section, the analyst specifies the trussTag, the two end nodes,
and the Section tag, secTag. 

The valid queries to a truss element when creating an ElementRecorder
are 'axialForce', 'stiff', 'material matArg1 matArg2 ..., 'section sectArg1
sectArg2 ...'. There will be more valid queries after the interface
for the methods involved have been further developed.

\subsubsection{The Elastic Beam Column Element}
{\sf\small
\begin{verbatim}
   element elasticBeamColumn eleTag? iNode? jNode? A? E? I? transfTag?
   element elasticBeamColumn eleTag? iNode? jNode? A? E? G? Jx? Iy? Iz? transfTag?
\end{verbatim}
}

\noindent The arguments to construct an elastic beam-column element depend on
the dimension of the problem, ndm. For a two dimensional problem the
analyst specifies the eleTag, the two end nodes, the section area,
Young's modulus and second moment of section area. For the three dimensional
problem the analyst specifies the eleTag, two end nodes, and the material and
section properties. Note that Iz (Iy) is the second moment of area about the beam section
local z-axis (y-axis) (Figure~\ref{localAxis}). For both two and three dimensional problems, 
the final argument to the elasticBeamColumn command is a transfTag, which identifies a 
previously defined CrdTransf object.

The valid queries to an elastic beam column element when creating an ElementRecorder
are 'stiffness' and 'force'.

\subsubsection{The Nonlinear Beam Column Element}
{\sf\small
\begin{verbatim}
   element nonlinearBeamColumn eleTag? iNode? jNode? numIntgrPts? secTag? transfTag?
                               <-mass massDens> <-iter maxIters tol> 
\end{verbatim}
}

\noindent The nonlinearBeamColumn element is based on the non-iterative flexibility
formulation, and considers the spread of plasticity along the
element. The arguments to construct the element are its tag, eleTag, 
the two end nodes, iNode and jNode, the number of integration points along the element,
numIntgrPts, the section tag, secTag (must be pre-defined), and the geometric
transformation tag, transfTag (pre-defined).
The integration along the element is based on Gauss-Lobatto quadrature
rule (two integration points at the element ends).  The element
is prismatic, i.e. the beam is represented by the section 
model identified by secTag at each integration point.  An element mass 
density per unit length, massDens, from which a lumped mass matrix is formed, is 
specified via the -mass switch.  An option is also
provided for the iterative form of the flexibility formulation by the -iter switch.
The arguments for the iterative form are maxIters, the maximum number
of iterations to undertake to satisfy element compatibility; and tol, the tolerance
for satisfaction of element compatibility.  Note that the iterative form can improve 
the rate of global convergence at the expense of more local element computation.

The valid queries to a nonlinearBeamColumn element when creating an ElementRecorder
are 'force', 'stiffness', or 'section secNum secArg1 secArg2 ...'.

\subsubsection{The Beam With Hinges Element}
{\sf\small
\begin{verbatim}
   element beamWithHinges eleTag? iNode? jNode? secTagI? ratioI? secTagJ? ratioJ?
                          E? A? I? transfTag? <-mass massDens> <-iter maxIters tol> 
   element beamWithHinges eleTag? iNode? jNode? secTagI? ratioI? secTagJ? ratioJ?
                          E? A? Iz? Iy? G? J? transfTag? <-mass massDens>
                          <-iter maxIters tol> 
\end{verbatim}
}

\noindent The beamWithHinges element is based on the non-iterative flexibility
formulation, and considers plasticity to be concentrated over specified hinge lengths
at the element ends.  The remaining beam interior is considered to be linear elastic. 
The arguments to construct the element are its tag, eleTag, 
the two end nodes, iNode and jNode, the section at node I, secTagI (pre-defined), the 
ratio of hinge length I to total element length, ratioI, the section at node J, secTagJ 
(pre-defined), the ratio of hinge length J to total element length, ratioJ, elastic axial stiffness, 
EA, elastic bending stiffness, EI, and the geometric transformation tag, transfTag (pre-defined).
For three-dimensional problems EIz (EIy) is the bending stiffness about the section local z-axis
(y-axis) and GJ is the torsional stiffness (Figure~\ref{localAxis}).  Note that these elastic properties are only
integrated over the beam interior. The integration over the hinge lengths is mid-point 
integration.  An element mass density per unit length, massDens, from which a lumped mass matrix
is formed, is specified via the -mass switch.  An option is also provided for the 
iterative form of the flexibility formulation by the -iter switch.
The arguments for the iterative form are maxIters, the maximum number
of iterations to undertake to satisfy element compatibility; and tol, the tolerance
for satisfaction of element compatibility.  Note that the iterative form can improve 
the rate of global convergence at the expense of more local element computation.

The valid queries to a beamWithHinges element when creating an ElementRecorder
are 'force', 'stiffness', 'rotation' (hinge rotation), or 'section secNum secArg1 secArg2 ...'.

\subsubsection{The Zero Length Element}
{\sf\small
\begin{verbatim}
   element zeroLength eleTag? iNode? jNode? -mat matTag1? matTag2? ...
                      -dir dir1? dir2? ... <-orient x1? x2? x3? yp1? yp2? yp3?>
\end{verbatim}
}

\noindent Constructs a zero length element defined by two nodes at the same geometric location.
The nodes are connected by multiple UniaxialMaterial objects to represent the force-deformation 
relationship for the element.  The zero length element is identified by its tag, eleTag, nodes 
iNode and jNode, UniaxialMaterial objects (previously defined) identified by matTag1 matTag2 ..., 
and material directions dir1 dir2 ... .  Two optional orientation vectors can be specified for the
element.  The vector x defines the local x-axis for the element and the vector yp lies in the local
x-y plane for the element.  The local z-axis is the cross product between x and yp, and the local
y-axis is the cross product between the local z-axis and x.  If these orientation vectors are not
specified, the local element axes coincide with the global axes.  The values for dir are 1 through 6,
where 1,2,3 indicate translation along the local x,y,z axes respectively; while 4,5,6 indicate
rotation about the local x,y,z axes respectively.

The valid queries to a zeroLength element when creating an ElementRecorder
are 'force', 'deformation', 'stiff', or 'material matNum matArg1 matArg2 ...'.

% \subsubsection{The Damper Element}
% {\sf\small
% \begin{verbatim}
%    element damper eleTag? iNode? jNode? matTag? <mass?>
% \end{verbatim}
% }

% \noindent Constructs a damper element. The damper element is identified by its tag,
% eleTag, end nodes iNode and jNode, and UniaxialMaterial object identified by matTag 
% (previously defined). The damper element can use any UniaxialMaterial object for its
% damping force-velocity relation. An optional total element mass can be specified.  This
% input element mass is divided and lumped at each node of the damper element.

% The valid queries to a damper element when creating an ElementRecorder
% are 'force', 'stiff', and 'material matArg1 matArg2 ...'.

\subsubsection{The Quad Element}
{\sf\small
\begin{verbatim}
   element quad eleTag? iNode? jNode? kNode? lNode? thick? type matTag? 
                <pressure? rho? b1? b2?>
\end{verbatim}
}

\noindent Constructs a FourNodeQuad element which uses the bilinear isoparametric formulation. 
The quad element is identified by its tag,
eleTag, the four nodes iNode, jNode, kNode, and lNode, the element thickness (constant), 
a string representing the material behavior, type, and an NDMaterial object identified 
by matTag (previously defined). Valid options for the parameter type depend on the NDMaterial
object and its available material formulations.  See the NDMaterial object documentation 
for valid types. A uniform element normal traction can be specified by the pressure
argument. Constant body forces b1 and b2, defined in the isoparametric domain, can be specified as
well and are optional, as is the element mass density per unit volume rho, for which
a lumped element mass matrix is computed. Consistent nodal loads are computed for the
pressure and body forces. The four nodes i through l must be input in counter-clockwise order 
around the element.

The valid queries to a quad element when creating an ElementRecorder
are 'force', 'stiffness', or 'material matNum matArg1 matArg2 ...',  where matNum
represents the material object at the integration point corresponding to the node numbers
in the isoparametric domain.

\subsection{The fix Command}
{\sf\small
\begin{verbatim}
   fix nodeTag? (ndf values?)
\end{verbatim}
}

To construct homogeneous single-point boundary constraints, the user
specifies the nodeTag of the node to be constrained and ndf (0,1)
values. A $1$ specified for the i'th value indicates that the dof for
the i'th degree-of-freedom is to be constrained, a $0$ that it is to be
left unconstrained.

\subsection{The pattern Command}
{\sf\small
\begin{verbatim}
   pattern patternType patternTag? <arguments for pattern type>
\end{verbatim}
}

The pattern command is used to construct a LoadPattern object, its
associated TimeSeries object and the Load and Constraint objects for
the pattern. There are two valid types of patternType: Plain and
UniformExcitation. 

{\sf\small
\begin{verbatim}
   pattern Plain patternTag? timeSeriesType <args for time series> { 
        load ...
        sp ...
   }	
\end{verbatim}
}

\noindent The string Plain is used to construct an ordinary
LoadPattern object with a unique tag among load patterns in the Domain
of patternTag.  The type of TimeSeries object associated with the
LoadPattern object depends on the timeSeriesType. At present 3 types
are recognized: Constant, Linear, Rectangular, Sine, and Series. The last 
argument in the command is a list of commands to create nodal load and single-point
constraints. The arguments for creating the TimeSeries object are as
follows: 

\paragraph{The Constant Time Series}

{\sf\small
\begin{verbatim}
   Constant <-factor cFactor?> 
\end{verbatim}
}

\noindent To associate with the LoadPattern object a TimeSeries object
of type ConstantSeries. A Constant causes a load factor
of cFactor to be applied to the loads and constraints in the pattern, independent of the
time in the domain. The default value of cFactor is $1.0$. The analyst
has the option of changing this using the optional arguments -factor and cFactor.

\paragraph{The Linear Time Series}

{\sf\small
\begin{verbatim}
   Linear <-factor cFactor?> 
\end{verbatim}
}

\noindent To associate with the LoadPattern object a TimeSeries object
of type LinearSeries. A LinearSeries causes a load factor
of cFactor * time to be applied to the loads and constraints in the pattern. The
default value of cFactor is $1.0$. The analyst has the option of
changing this using the optional arguments -factor and cFactor.

\paragraph{The Rectangular Time Series}

{\sf\small
\begin{verbatim}
   Rectangular tStart? tFinish? <-factor cFactor?> 
\end{verbatim}
}

\noindent To associate with the LoadPattern object a TimeSeries object
of type RectangularSeries. A RectangularSeries causes a load factor of
cFactor * time to be applied to the loads and constraints in the pattern when the
value of time is between tStart and tFinish. The default value of
cFactor is $1.0$. The analyst has the option of changing this using
the optional arguments -factor and cFactor. 

\paragraph{The Sine Time Series}

{\sf\small
\begin{verbatim}
   Sine tStart? tFinish? T? <-shift shift?> <-factor cFactor?> 
\end{verbatim}
}

\noindent To associate with the LoadPattern object a TimeSeries object
of type TrigSeries. A TrigSeries is defined by a period, T, and optional
phase shift (radians).  The TrigSeries causes a load factor of
cFactor * $\sin(\frac{2\pi*(time-tStart)}{T} + shift)$ 
to be applied to the loads and constraints in the pattern when the
value of time is between tStart and tFinish. The default value of shift
is $0.0$ and the default value of cFactor is $1.0$. The analyst has the 
option of changing these values using the optional switches -shift and -factor.

\paragraph{The Path Time Series}

{\sf\small
\begin{verbatim}
   Series -dt dt? -values {list of points} <-factor cFactor?> 
   Series -time {list of times}  -values {list of points} <-factor cFactor?> 
   Series -dt dt? -filePath fileName? <-factor cFactor?> 
   Series -fileTime fileName1? -filePath fileName2? <-factor cFactor?> 
\end{verbatim}
}


\noindent To associate with the LoadPattern object a TimeSeries object
of type PathSeries or PathTimeSeries (if time increments not
constant). A PathSeries causes a load factor of cFactor * 
value to be applied to the loads and constraints in the pattern. The value used
depends on the time and a linear interpolation between points on the
load path. There are a number of ways to specify the load path, for a load
path where the points are specified at constant time intervals the
'-dt' option. Following dt is either the option '-values' indicating the
points are in the accompanying list enclosed in braces, or '-filePath' indicating that the
points are contained in the file given by fileName. For a load path
where the points are specified at non-constant time intervals the
analyst can provide the time increments and points at these intervals
in two strings or two files (not a string and a file). The default value of
cFactor is $1.0$. The analyst has the option of changing this
using the optional arguments -factor and cFactor.

\paragraph{The UniformExcitation Pattern}

{\sf\small
\begin{verbatim}
   pattern UniformExcitation patternTag? dir? factor? -accel fileName? dt?
\end{verbatim}
}

\noindent A LoadPattern representing a UniformExcitation is created
using this command.  The pattern has a tag, patternTag, unique among
all load patterns.  A UniformExcitation is implemented through a
PathSeries object which reads acceleration values contained in the
file represented by fileName.  These values are assumed to be spaced
at a constant time interval of dt and the factor is applied to each.
The UniformExcitation associates these values to all fixed degrees of
freedom in the direction dir (1,2, or 3) when formulating inertial loads for a
transient analysis.

\subsubsection{The load Command}
{\sf\small
\begin{verbatim}
   load nodeTag? (ndf values?) <-const> <-pattern patternTag?>
\end{verbatim}
}

The load command is used to construct a NodalLoad object. The first
argument to the load command, nodeTag, is an integer tag identifying the
node on which the load acts. Following the tag is the ndf reference
load values that are to be applied to the node. The nodal load is
added to the LoadPattern being defined in the enclosing scope of the
pattern command. Optional arguments are the string -const, which
identifies the load being applied to the node as being independent of
any load factor, and the string -pattern and an integer patternTag
identifying the load pattern to which the load is to be added.

\subsubsection{The sp Command}
{\sf\small
\begin{verbatim}
   sp nodeTag? dofTag? value? <-const> <-pattern patternTag?>
\end{verbatim}
}

The sp command is used to construct a SP\_Constraint object. The first
argument to the sp command, nodeTag, is an integer tag identifying the
node on which the single-point constraint acts. Following this tag is 
an integer, dofTag, identifying the degree-of-freedom at the node being
constrained, valid range is 1 through ndf. (Note: fortran indexing at
the interpreter, internally OpenSees uses C indexing). The third argument,
value, specifies the reference value of the constraint. The constraint is
added to the LoadPattern being defined in the enclosing scope of the
pattern command. Optional arguments are the string -const, which
identifies the constraint being applied to the node as being independent of
any load factor, and the string -pattern and an integer patternTag
identifying the load pattern to which the constraint is to be added.

\subsubsection{The equalDOF Command}
{\sf\small
\begin{verbatim}
   equalDOF rNodeTag? cNodeTag? dof1? dof2? ...
\end{verbatim}
}

The equalDOF command is used to construct a multi-point constraint between
the nodes identified by rNodeTag and cNodeTag.  rNodeTag is the retained, or
master node, and cNodeTag is the constrained, or slave node.  dof1 dof2 ... 
represent the nodal degrees of freedom that are constrained at the cNode to
be the same as those at the rNode. The valid range for dof1 dof2 ... is 1
through ndf, the number of nodal degrees of freedom.

\subsubsection{The rigidDiaphragm Command}
{\sf\small
\begin{verbatim}
   rigidDiaphragm perpDirn? masterNodeTag? slaveNodeTag1 ...
\end{verbatim}
}

The rigidDiaphragm command is used to construct a number of
MP\_Constraint objects. These constraints will constrain certain
degrees-of-freedom at the the slave nodes listed to move as if in a
rigid plane with the master node. The rigid plane can be the 12, 13 or
23 planes. The rigid plane is specified by the user providing the
perpendicular plane direction, ie 3 for 12 plane. The constraint
object is constructed assuming small rotations. The rigidDiaphragm
command works only for problems in three dimesions with six
degrees-of-freedom at the nodes. 

\subsubsection{The rigidLink Command}
{\sf\small
\begin{verbatim}
   rigidLink -type? masterNodeTag? slaveNodeTag
\end{verbatim}
}

The rigidLink command is used to construct a single MP\_Constraint object.
The type can be either rod or beam. If rod is specified the
translational degrees-of-freedom will be constrained to be exactly the
same as those at the master node. If beam is specified a rigid beam
constraint is imposed on the slave node, that is the translational and
rotational degrees of freedom are constrained. The constraint object
constructed for the beam option assumes small rotations.

\section {The analysis Command}

{\sf\small
\begin{verbatim}
   analysis analysisType 
\end{verbatim}
}

The analysis command is used to construct the Analysis object.
The valid strings for analysisType are: Static, Transient. 

\subsection{Static Analysis}

{\sf\small
\begin{verbatim}
   analysis Static 
\end{verbatim}
}

To construct a StaticAnalysis object. The analysis object is
constructed with the component objects, i.e. SolutionAlgorithm, StaticIntegrator,
ConstraintHandler, DOF\_Numberer, LinearSOE, and LinearSolver objects
previously created and by the analyst. If none has been created,
default objects are constructed and used. These defaults are a
NewtonRaphson EquiSolnAlgo with a CTestNormUnbalance with a tolerance
of $1.0e-6$ and a maximum of $25$ iterations, a PlainHandler
ConstraintHandler, an RCM DOF\_Numberer, a LoadControl
StaticIntegrator with a constant load increment of $1.0$, and a
profiled symmetric positive definite LinearSOE and LinearSolver.

\subsection{Transient Analysis}

{\sf\small
\begin{verbatim}
   analysis Transient
\end{verbatim}
}

To construct a DircetIntegrationAnalysis object. The analysis object is
constructed with the component objects, i.e. SolutionAlgorithm,
TransientIntegrator, ConstraintHandler, DOF\_Numberer, LinearSOE, and
LinearSolver objects previously created by the analyst. If none has been  
created, default objects are constructed and used. These defaults are
a NewtonRaphson EquiSolnAlgo with a CTestNormUnbalance with a
tolerance of $1.0e-6$ and a maximum of $25$ iterations, a PlainHandler
ConstraintHandler, an RCM DOF\_Numberer, a Newmark
TransientIntegrator with $\gamma = 0.5$ and $\beta = 0.25$, and a profiled
symmetric positive definite LinearSOE and LinearSolver.

\section {The constraints Command}
{\sf\small
\begin{verbatim}
   constraints constraintHandlerType <args for handler type>
\end{verbatim}
}

The constraints command is used to construct the ConstraintHandler
object. The ConstraintHandler object determines how the constraint
equations are enforced in the analysis. The valid strings for
constraintHandlerType are: Plain, Penalty, Lagrange, and Transformation.

\subsection{Plain Constraints}

{\sf\small
\begin{verbatim}
   constraints Plain
\end{verbatim}
}

\noindent This will create a PlainHandler which is only capable of enforcing
homogeneous single-point constraints. If other types of constraints
exist in the domain, a different constraint handler must be specified.

\subsection{Penalty Method}

{\sf\small
\begin{verbatim}
   constraints Penalty alphaSP? alphaMP?
\end{verbatim}
}

\noindent To construct a PenaltyConstraintHandler which will cause the
constraints to be enforced using the penalty method. The alphaSP and
alphaMP values are the factors used when adding the single-point and
multi-point constraints into the system of equations.

\subsection{Lagrange Multipliers}

{\sf\small
\begin{verbatim}
   constraints Lagrange <alphaSP?> <alphaMP?>
\end{verbatim}
}

\noindent To construct a LagrangeConstraintHandler which will cause the
constraints to be enforced using the method of Lagrange multipliers. The alphaSP and
alphaMP values are the factors used when adding the single-point and
multi-point constraints into the system of equations. If no values are specified
values of $1.0$ are assumed. Values other than $1.0$ are permited to offset numerical
roundoff problems. It should be
noted that the resulting system of equations are not positive definite
due to the introduction of zeroes on the diagonal by the constraint equations.

\subsection{Transformation Method}

{\sf\small
\begin{verbatim}
   constraints Transformation 
\end{verbatim}
}

\noindent To construct a TransformationConstraintHandler which will cause the
constraints to be enforced using the transformation method. It should be noted
that no retained node in an MP\_Constraint can also be specified as being a
constrained node in another MP\_Constraint with the current implementation.

\section {The integrator Command}
{\sf\small
\begin{verbatim}
   integrator integratorType <args for integrator type>
\end{verbatim}
}

The integrator command is used to construct the Integrator
object. The Integrator object determines the meaning of the terms in
the system of equation object. The valid strings for
integratorType for a static analysis are: LoadControl,
DisplacementControl, MinUnbalDispNorm, Arclength, and ArcLength1.
{\bf It should be noted that static
integrators should only be used with a Linear TimeSeries object with a
factor of $1.0$}. The valid strings for integratorType are for dynamic
analysis: Newmark, Newmark1, HHT, and HHT1

\subsection{Load Control}

{\sf\small
\begin{verbatim}
   integrator LoadControl dlambda1? Jd? minLambda? maxLambda?
\end{verbatim}
}

To construct a StaticIntegrator object of type LoadControl. The third
argument, dlambda1, is a floating-point number specifying the first
load increment (pseudo-time step) in the next invocation of the
analysis command. The load increment at subsequent iterations $i$ is
related to the load increment at (i-1), dlambda(i-1), and the number
of iterations at $i-1$, J(i-1), by the following: dLambda(i) =
dlambda(i-1)*Jd/J(i-1). The floating-point arguments minLambda and
maxLambda are used to bound the increment.

\subsection{Displacement Control}

{\sf\small
\begin{verbatim}
   integrator  DisplacementControl nodeTag? dofTag? dU1? Jd? minDU? maxDU?
\end{verbatim}
}

To construct a StaticIntegrator object of type DisplacementControl. The third
and fourth arguments, nodeTag and dofTag, are integers identifying the
node and the degree-of-freedom at the node (1,ndf), whose response
controls the solution. The fourth argument dU1, is a floating-point
number specifying the first displacement increment (pseudo-time step)
in the next invocation of the analysis command. The displacement
increment at subsequent iterations $i$ is related to the displacement
increment at (i-1), dU(i-1), and the number of iterations at $i-1$,
J(i-1), by the following: dU(i) = dU(i-1)*Jd/J(i-1). The
floating-point arguments minDU and maxDU are used to bound the
increment.

\subsection{Minimum Unbalanced Displacement Norm}

{\sf\small
\begin{verbatim}
   integrator MinUnbalDispNorm dlambda11? Jd? minLambda? maxLambda?
\end{verbatim}
}

To construct a StaticIntegrator object of type MinUnbalDispNorm. The third
argument, dlambda11, is a floating-point number specifying the first
load increment (pseudo-time step) at the first iteration in the next
invocation of the analysis command. The first load increment at subsequent
iterations $i$ is related to the load increment at (i-1),
dlambda(i-1), and the number of iterations at $i-1$, J(i-1), by the
following: dLambda(1i) = dlambda(i-1)*Jd/J(i-1). The floating-point
arguments minLambda and maxLambda are used to bound the increment.

\subsection{Arc-Length Control}

{\sf\small
\begin{verbatim}
   integrator ArcLength arclength? alpha?
   integrator ArcLength1 arclength? alpha?
\end{verbatim}
}

To construct a StaticIntegrator object of type ArcLength or
ArcLength1. The third and fourth arguments, are floating-point numbers
defining the two arc length parameters to be used.

\subsection{Newmark Method}

{\sf\small
\begin{verbatim}
   integrator Newmark gamma? beta? <alphaM? betaK? betaKinit? betaKcomm?> 
   integrator Newmark1 gamma? beta? <alphaM? betaK? betaKinit? betaKcomm?> 
\end{verbatim}
}

To construct a TransientIntegrator object of type Newmark or Newmark1,
(Newmark1 predicts displacement and velocity and sets acceleration to
$0$, whereas Newmark predicts velocity and acceleration and leaves
displacement as is). The third and fourth arguments, are
floating-point numbers defining the two Newmark parameters $\gamma$ and
$\beta$. Optional arguments are provided for Rayleigh damping, where
the damping matrix D = alphaM * M + betaK * Kcurrent + betaKcomm *
KlastCommit + betaKinit * Kinit.

\subsection{Hilbert-Hughes-Taylor Method}

{\sf\small
\begin{verbatim}
   integrator HHT alpha? <alphaM? betaK? betaKinit? betaKcomm?> 
   integrator HHT1 alpha? <alphaM? betaK? betaKinit? betaKcomm?> 
\end{verbatim}
}

To construct a TransientIntegrator object of type HHT or HHT1,
(HHT1 predicts displacement and velocity and sets acceleration to
$0$, whereas HHT predicts velocity and acceleration and leaves
displacement as is). The third and fourth arguments, are
floating-point numbers defining the two Newmark parameters
$\alpha$. Optional arguments are provided for Rayleigh damping, where
the damping matrix D = alphaM * M + betaK * Kcurrent + betaKcomm *
KlastCommit + betaKinit * Kinit.

\section {The algorithm Command}
{\sf\small
\begin{verbatim}
   algorithm algorithmType <args for algorithm type>
\end{verbatim}
}

The algorithm command is used to construct the Algorithm
object. The Algorithm object determines the solution of steps taken
to solve the non-linear equation. The valid strings for algorithmType
are: Linear, Newton, ModifiedNewton. None of the present types
require additional arguments.

\subsection {Linear Algorithm}
{\sf\small
\begin{verbatim}
   algorithm Linear
\end{verbatim}
}

To construct a Linear algorithm object which takes one iteration to solve
the system of equations.

\subsection {Newton Algorithm}
{\sf\small
\begin{verbatim}
   algorithm Newton
\end{verbatim}
}

To construct a NewtonRaphson algorithm object which uses the Newton-Raphson
method to advance to the next time step. The tangent is updated at each iteration.

\subsection {Modified Newton Algorithm}
{\sf\small
\begin{verbatim}
   algorithm ModifiedNewton
\end{verbatim}
}

To construct a ModifiedNewton algorithm object which uses the modified Newton-Raphson
method to advance to the next time step. The tangent at the first iteration of the current
time step is used to iterate to the next time step.

\section {The test Command}

{\sf\small
\begin{verbatim}
   test convergenceTestType <args for test type>
\end{verbatim}
}

Certain SolutionAlgorithm objects require a ConvergenceTest object to
determine if convergence has been achieved at the end of an iteration
step. The test command is used to construct ConvergenceTest object. The
valid strings for convergenceTestType are NormUnbalance, NormDispIncr
and EnergyIncr.

\subsection{Norm Unbalance Test}

{\sf\small
\begin{verbatim}
   test NormUnbalance tol? maxNumIter? <printFlag?>
\end{verbatim}
}

\noindent To construct a CTestNormUnbalance which tests positive for
convergence if the 2-norm of the $b$ vector (the unbalance) in the
LinearSOE object is less than tol. A maximum of maxNumIter iterations
will be performed before failure to converge will be returned. The
optional printFlag can be used to print information on convergence, a
$1$ will print information on each step, a $2$ when convergence has
been achieved.

\subsection{Norm Displacement Increment Test}

{\sf\small
\begin{verbatim}
   test NormDispIncr tol? maxNumIter? <printFlag?>
\end{verbatim}
}

\noindent To construct a CTestNormDispIncr which tests positive for
convergence if the 2-norm of the $x$ vector (the displacement increments)
in the LinearSOE object is less than tol. A maximum of maxNumIter iterations
will be performed before failure to converge will be returned. The
optional printFlag can be used to print information on convergence, a
$1$ will print information on each step, a $2$ when convergence has
been achieved.

\subsection{Energy Increment Test}

{\sf\small
\begin{verbatim}
   test EnergyIncr tol? maxNumIter? <printFlag?>
\end{verbatim}
}

\noindent To construct a CTestEnergyIncr which tests positive for
convergence if the half the inner-product of the $x$ and $b$ vectors
(displacement increments and unbalance) in the LinearSOE object is less 
than tol. A maximum of maxNumIter iterations will be performed before failure 
to converge will be returned. The optional printFlag can be used to print
information on convergence, a $1$ will print information on each step,
a $2$ when convergence has been achieved.


\section {The numberer Command}
{\sf\small
\begin{verbatim}
   numberer numbererType <args for numberer type>
\end{verbatim}
}

The numberer command is used to construct the DOF\_Numberer
object. The DOF\_Numberer object determines the mapping between
equation numbers and degrees-of-freedom.  The valid strings are
are: Plain, RCM. None of the present types require additional
arguments. As certain system of equation and solver objects do
their own mapping, i.e. SuperLU, UmfPack, Kincho's specifying a
numberer other than plain may be a waste of time.

\section {The system Command}
{\sf\small
\begin{verbatim}
   system systemType <args for system type>
\end{verbatim}
}

The system command is used to construct the LinearSOE and a LinearSolver
objects to store and solve the system of equations in the
analysis. The valid types of systemType commands are: BandGeneral,
BandSPD, ProfileSPD, SparseGeneral, UmfPack, and SparseSPD.

\paragraph{The BandGeneral SOE}

{\sf\small
\begin{verbatim}
   system BandGeneral
\end{verbatim}
}

\noindent To construct an un-symmetric banded system of equation object which will
be factored and solved during the analysis using the Lapack band
general solver.

\paragraph{The BandSPD SOE}

{\sf\small
\begin{verbatim}
   system BandSPD
\end{verbatim}
}

\noindent To construct a symmetric positive definite banded system of equation
object which will be factored and solved during the analysis using the
lapack band spd solver.

\paragraph{The ProfileSPD SOE}

{\sf\small
\begin{verbatim}
   system ProfileSPD
\end{verbatim}
}

\noindent To construct a symmetric positive definite profile system of equation
object which will be factored and solved during the analysis using a
profile solver.

\paragraph{The SparseGeneral SOE}

{\sf\small
\begin{verbatim}
   system SparseGeneral <-piv>
\end{verbatim}
}

\noindent To construct a general sparse system of equation
object which will be factored and solved during the analysis using the
SuperLU solver. By default no partial pivoting is performed. The
analyst can change this by specifying the -piv option.

\paragraph{The UmfPack SOE}

{\sf\small
\begin{verbatim}
   system UmfPack 
\end{verbatim}
}

\noindent To construct a general sparse system of equation
object which will be factored and solved during the analysis using the
UMFPACK solver. \\

\paragraph{The SparseSPD SOE}

{\sf\small
\begin{verbatim}
   system SparseSPD
\end{verbatim}
}

\noindent To construct a sparse symmetic positive definite system of
equation object which will be factored and solved during the analysis
using a sparse solver developed at Stanford by Kincho Law.

\section {The recorder Command}

{\sf\small
\begin{verbatim}
   recorder recorderType <args for type>
\end{verbatim}
}

The recorder command is used to construct a Recorder object. A
Recorder object is used to monitor items of interest to the analyst at
each commit(). Valid strings for recorderType are MaxNodeDisp, Element
Node, display. Note that display is not yet available on the
Windows version of opensees, this will be fixed shortly.

\subsection{The MaxNodeDisp Recorder}

{\sf\small
\begin{verbatim}
   recorder MaxNodeDisp dof node1? node2? ...
\end{verbatim}
}

\noindent To construct a recorder of type MaxNodeDisp to record the values of
the maximum absolute values of the displacement in the
degree-of-freedom direction dof for the nodes node1, node2, ...

\subsection{The Node Recorder}

{\sf\small
\begin{verbatim}
   recorder Node fileName responseType <-time> -node node1? ... -dof dof1? ...
\end{verbatim}
}

\noindent To construct a recorder of type NodeRecorder to record the 
responseType in the degree-of-freedom directions dof1, ..., at
the nodes node1, ... The results are saved in the file given
by the string fileName. Each line of the file contains the result for
a committed state of the domain. An optional argument -time will place
the pseudo time of the Domain as the first entry in the line. The
responseType defines the response type to be recorded, limited to one
of the following: disp, vel, accel and incrDisp for displacements,
velocities, accelerations and incremental displacements. 

\subsection{The Element Recorder}

{\sf\small
\begin{verbatim}
   recorder Element eleID1? ...  <-file fileName> <-time> arg1? arg2? ...
\end{verbatim}
}

\noindent To construct a recorder of type ElementRecorder to record
the response at a number of elements with tag eleID1, ... . The
response recorded is element dependent and depends on the arguments
arg1 arg2 ..., which are passed to the setResponse() element
method. The results are printed directly to the screen or are saved in
a file fileName if the optional -file argument is provided. An
optional argument -time will place the pseudo time of the Domain as
the first entry in the line. Note on playback, unless the results are
stored in a file, nothing will be printed to the secreen for this type
of recorder. 

\subsection{The Display Recorder}

{\sf\small
\begin{verbatim}
   recorder display windowTitle? xLoc? yLoc? xPixels? yPixels? <-file fineName?>
\end{verbatim}
}

\noindent To open a graphical window with the title windowTitle at location
xLoc, yLoc which is xPixels wide by yPixels high for the
displaying of graphical information. A TclFeViewer object is
constructed. This constructor adds a number of additional commands to
opensees, similar to the construction of the BasicBuilder. These other
commands are used to define the viewing system for the image that is
placed on the screen. These commands are under review and will be
discussed in the next version of this document. If the optional -file
argument is provided, in addition to displaying the model in the window,
information is sent to a file so that the images displayed may be
redisplayed at a later time. 

\subsection{The Plot Recorder}

{\sf\small
\begin{verbatim}
   recorder plot fileName? windowTitle? xLoc? yLoc? xPixels? yPixels? 
        <-columns xCol? yCol?>
\end{verbatim}
}

\noindent To open a graphical window with the title windowTitle at location
xLoc, yLoc which is xPixels wide by yPixels high for the plotting the contents of the
file fileName. Unless the optional -columns flag is passed, the first column of the file
is used as the x-axis and the second column as the y-axis.

\section {The analyze Command}
{\sf\small
\begin{verbatim}
   analyze numIncr? <dt?>
\end{verbatim}
}

\noindent To invoke analyze(numIncr, $<$dt$>$) on the Analysis object constructed with the
analysis command. 



\section {The eigen Command}
{\sf\small
\begin{verbatim}
   eigen numEigenvalues?
\end{verbatim}
}

\noindent To perform a generalized eigenvalue problem to determine the
first numEigenvalues eigenvalues and eigenvectors. The eigenvectors
are stored at the nodes and can be printed out. Currently each
invocation of this command constructs a new EigenvalueAnalysis object,
each with new component objects: a ConstraintHandler of type Plain, an
EigenvalueSOE and solver of type BandArpackSOE and BandArpackSolver
and an algorithm of type FrequencyAlgo. These objects are destroyed
when the command has finished. This will change.

\section {The database Commands}
{\sf\small
\begin{verbatim}
   database databaseType 
\end{verbatim}
}


\noindent To construct a FE\_Datastore object of type
databaseType. Currently there is only one type of datastore object
available, that of type FileDatastore. The invocation of this command
will add the additional commands {\bf save} and {\bf restore} to the
opensees interpreter to allow users to save and restore model states.

{\sf\small
\begin{verbatim}
   database File fileName
\end{verbatim}
} 

\noindent To construct a datastore object of type FileDatastore. The
FileDatastore object will save the data into a number of files, e.g
fileName.id11 for all ID objects of size 11 that sendSelf() is invoked
with.

\subsection {The save Command}
{\sf\small
\begin{verbatim}
   save commitTag?
\end{verbatim}
} 

\noindent To save the state of the model in the database. The
commitTag is the unique identifier that can be used to restore the
state at a latter time.


\subsection {The restore Command}
{\sf\small
\begin{verbatim}
   restore commitTag?
\end{verbatim}
} 

\noindent To restore the state of the model from the information
stored in the database. The state of the model will be the same as 
when the command save commitTag was invoked.



\section {Misc. Commands}

\paragraph {The playback Command}

{\sf\small
\begin{verbatim}
   playback commitTag?
\end{verbatim}
}

\noindent To invoke playback on all Recorder objects constructed with the record
command. The record() method is invoked with the integer commitTag.

\paragraph {The print Command}

{\sf\small
\begin{verbatim}
   print <fileName> 
   print <fileName> -node <-flag flag?> <node1? node2? ..>
   print <fileName> -ele <-flag flag?> <ele1? ele2? ..>
\end{verbatim}
}

\noindent To cause output to be printed to a file or stderr, if fileName is not
specified. If no string qualifier is used, the print method is
invoked on all objects of the domain. If the string -node or -ele is provided,
the print() method is invoked on just the nodes or elements. With
the analyst can send the integer flag to the print() method. The
analyst with these can also limit the element and nodes on which the
print() method is invoked by supplying the objects tags.

\paragraph {The reset Command}

{\sf\small
\begin{verbatim}
   reset
\end{verbatim}
 }

\noindent To set the state of the domain to its original state. Invokes
revertToStart() on the Domain object.

\paragraph {The wipe Command}

{\sf\small
\begin{verbatim}
   wipe
\end{verbatim}
}

\noindent To destroy all objects constructed, i.e. to start over again
without having to exit and restart the interpreter.

\paragraph {The wipeAnalysis Command}

{\sf\small
\begin{verbatim}
   wipeAnalysis
\end{verbatim}
}

\noindent To destroy all objects constructed for the analysis in order
to start a new type of analysis.

\paragraph {The loadConst Command}

{\sf\small
\begin{verbatim}
   loadConst <-time pseudoTime?>
\end{verbatim}
}

\noindent To invoke setLoadConst() on all LoadPattern objectswhich
have been created to this point. If the optional string -time or
is specified, the pseudo time in the domain will be set to pseudoTime.

\paragraph {The time Command}

{\sf\small
\begin{verbatim}
   time pseudoTime?
\end{verbatim}
}

\noindent To set the pseudo time in the domain to pseudoTime?

\paragraph {The build Command}

{\sf\small
\begin{verbatim}
    build
\end{verbatim}
}

\noindent To invoke build() on the ModelBuilder object. Has no effect
on a BasicBuilder object, but will on other types of ModelBuilder objects.

\paragraph {The video Command}

{\sf\small
\begin{verbatim}
    video -file fileName -window windowName?
\end{verbatim}
}

\noindent To construct a TclVideoPlayer object for displaying the images in a file
created by the recorder display command. The images are displayed by invoking the 
command {\em play}.


\section{Example}

\begin{figure}[htpb]
\begin{center}
\leavevmode
\hbox{%
%\epsfxsize=6.0in
%\epsfysize=4.2in
\epsffile{./Example.eps}}
\end{center}
\caption{Example 1}
\label{example1}
\end{figure}

\noindent An example openseesTcl script for the static analysis of the
simple linear three bar truss example shown in figure~\ref{example1} is
given in {\sf OpenSees/examples/tcl/example1.tcl}. First, the analyst creates a
ModelBuilder object. In this example a BasicBuilder object is
created for a two dimensional problem with two degrees-of-freedom at each
node. The construction of the BasicBuilder object adds new
commands to the interpreter, i.e. node, material, element, fix and
load. It is these commands which can be used by the analyst to
construct the model. 

{\sf\small \begin{verbatim}
# create the ModelBuilder object
model BasicBuilder -ndm 2 -ndf 2
\end{verbatim} }

The analyst then constructs the model. This is done by creating
the four Node objects, a Material object, three Element objects, some
Constraint objects, and finally a Load Pattern object with a
LinearSeries object containing a single NodalLoad object. 

{\sf\small \begin{verbatim}
# node nodeTag xLoc yLoc
node 1   0   0 
node 2 144   0 
node 3 168   0
node 4  72  96

# material matTag type <type args>
uniaxialMaterial Elastic 1 3000

# truss trussTag iNodeTag jNodeTag Area matTag
element truss 1 1 4 10 1
element truss 2 2 4  5 1
element truss 3 3 4  5 1

# constraint nodeTag xFix? yFix?
fix 1 1 1
fix 2 1 1
fix 3 1 1

# define a load pattern with a load
pattern Plain 1 Linear {
  # load nodeTag xForce yForce
  load 4 100 -50
}
\end{verbatim} }

After the model has been defined, the analyst then constructs the
Analysis using commands which already exist in opensees. This is done by
first constructing the components of the Analysis object. In this
example a BandSPD linear system of equation and a lapack solver
(default for BandSPD), a ConstraintHandler object which deals with
homogeneous single point constraints, an Integrator object of type
LoadControl with a load step increment of one, an Algorithm object of
type Linear, and a DOF\_Numberer object of type RCM (reverse
Cuthill-Mckee). Once these objects have been created, the Analysis
object is constructed. In this case a StaticAnalysis object.

{\sf\small \begin{verbatim}
# build the components for the analysis object
system BandSPD
constraints Plain
integrator LoadControl 1 1 1 1
algorithm Linear
numberer RCM

# create the analysis object 
analysis Static 
\end{verbatim} }

After the Analysis object is constructed a Recorder object is
created. In this example we create a NodeRecorder to record the
load factor and the two nodal displacements at Node 4, the results are
stored in the file {\sf /tmp/example.out}.

{\sf\small \begin{verbatim}
# create a Recorder object for the nodal displacements at node 4
recorder Node /tmp/example.out disp -load -node 4 -dof 1 2
\end{verbatim} }

Finally the analysis is performed and the results are printed. 

{\sf\small \begin{verbatim}
# perform the analysis
analyze 1

# print the results at node 4 and at all elements
print node 4
print ele
playback 1
\end{verbatim} }

When opensees is run and the commands outlined above are input by the analyst at the
interpreter prompt, or are sourced from a file using the {\sf source
filename} command, the following is output by the program. 

{\sf\small
\begin{verbatim}
Node: 4
        Coordinates  : 72 96 
        commitDisps: 0.530093 -0.177894 
        unbalanced Load: 100 -50 

Element: 1 type: Truss  iNode: 1 jNode: 4 Area: 10 
         strain: 0.00146451 axial load: 43.9352 
         unbalanced load: 26.3611 35.1482 -26.3611 -35.1482 
         Material: ElasticMaterialModel: 1  E: 3000

Element: 2 type: Truss  iNode: 2 jNode: 4 Area: 5 
         strain: -0.00383642 axial load: -57.5463 
         unbalanced load: 34.5278 -46.0371 -34.5278 46.0371 
         Material: ElasticMaterialModel: 1  E: 3000

Element: 3 type: Truss  iNode: 3 jNode: 4 Area: 5 
         strain: -0.00368743 axial load: -55.3114 
         unbalanced load: 39.1111 -39.1111 -39.1111 39.1111 
         Material: ElasticMaterialModel: 1  E: 3000

1 0.530093 -0.177894
\end{verbatim}
}


\end{document}





